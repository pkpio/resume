% !TEX program = XeLaTeX
\documentclass[]{friggeri-cv}

\begin{document}
\header{Praveen Kumar }{Pendyala}{Android developer | Linux enthusiast | Web hobbyist}

% In the aside, each new line forces a line break
\begin{aside}
  \section{Info}
    \noindent\rule{3cm}{0.5pt}
    %DOB: 26.10.1992
    \href{http://pkp.io}{http://pkp.io}
    \href{https://github.com/praveendath92}{github/praveendath92}    
  \section{Contact}
    \href{mailto:mail@pkp.io}{mail@pkp.io}
    (+49) 176-833-55188
    (+91) 882-882-9765
    ~
    Moltkestr. 15
    64295 Darmstadt
    Germany
    ~
  \section{Skills}
    \noindent\rule{3cm}{0.5pt}
  \section{Programming}
    Java, PHP, Kernel C, Python, JavaScript, Octave, Verilog
  \section{Web}
    AngularJS, HTML5, REST APIs, Bootstrap, MVCs, Bower
  \section{Android}
    Retrofit, Volley, Picasso, Material Design, Play Services
  \section{Tools}
    Git, SVN, MySQL, Asterisk, Xilinx ISE, Jekyll, \LaTeX, Moodle
  \section{Platforms}
    Linux, Windows, Mac OSX, Beaglebone, Google App Engine
  \section{Note}
    \noindent\rule{3cm}{0.5pt}
    Only included the most recent and major accomplishments in each section. For complete list, please refer to the web links in info section.
\end{aside}

\section{Education}
\begin{entrylist}
  \entry
    {2014-2016}
    {Masters, {\normalfont Distributed Software Systems}}
    {Technische Universität Darmstadt}
    {\emph{GPA of 3.5 out of 4.0 in US grading -- converted from 1.50 in German grading.}}
  \entry
    {2010–2014}
    {Bachelors, {\normalfont Electrical Engineering}}
    {Indian Institute of Technology Bombay}
    {}%GPA: 7.49 / 10}
\end{entrylist}

\section{Experience}
\begin{entrylist}
  \entry
    {since 2014}
    {Systems Security Lab, TU Darmstadt}
    {Android Developer and Security Researcher}
    {\emph{Working in Cloud, Android and Mobile security.}}
%  \entry
%    {05–08 2015}
%    {Google Summer of Code}
%    {Supporting mentor}
%    {\emph{Linux drivers for remote access from Android.}}
  \entry
    {05–08 2014}
    {Google Summer of Code}
    {Android Developer and Linux Kernel Programmer}
    {\emph{Developed kernel display drivers for remotely accessing Linux from Android.}}
  \entry
    {05–07 2013}
    {Rice University, Houston}
    {Hardware Security Researcher and Verilog Programmer}
    {\emph{Implemented a novel Hardware security module and published it in DAC 2014.}}
\end{entrylist}

\section{Publications}
\begin{entrylist}
  \entry
    {2017}
    {Phonion: Practical Protection of Metadata in Telephony Networks}
    {Proceedings on Privacy Enhancing Technologies (PoPETs) 2017}
    {}
    %{\emph{Stephan Heuser, Bradley Reaves, Praveen Kumar Pendyala, Henry Carter, Alexandra Dmitrienko, Negar Negar, William Enck, Ahmad-Reza Sadeghi and Patrick Traynor}}  
  \entry
    {2016}
    {DroidAuditor: Forensic Analysis of Application-Layer Privilege Escalation Attacks on Android}
    {Financial Cryptography and Data Security Conference 2016}
    {}
    %{\emph{Stephan Heuser, Marco Negro, Praveen Kumar Pendyala, and Ahmad-Reza Sadeghi}}   
%  \entry
%    {2014}
%    {PUFatt: Embedded Platform Attestation Based on Novel Processor-Based PUFs}
%    {Best paper candidate, 51st Design Automation Conference}
%    {\emph{Kong, J., F. Koushanfar, P. K. Pendyala, A. - R. Sadeghi, and C. Wachsmann}}

\end{entrylist}

\section{Achievements}
\begin{itemize}
  \item Winners of MAPPING Competition and \euro 20,000 prize money, CeBIT 2016, Hannover
  \item A. Richard Newton Young Student Fellow Award, DAC 2014, San Francisco
  \item Recipient of Indian Association for Research in Computing Science grant, 2013
  %\item All India category Rank 1 among 470,000 students in IIT JEE 2010
\end{itemize}
~

\section{Projects}
\begin{entrylist}
  \entry
    {2015–2016}
    {Omnishare}
    {Android, Cryptography, Dropbox and Google Drive APIs}
    {\emph{Developed an Android application for client side encryption of data before uploading to cloud services like Dropbox. The system also supports simultaneous multiple device installation and secure file sharing.}}
  \entry
    {2015–2016}
    {Phonion}
    {Python, Java, Twilio, Asterisk, Google Voice, Selenium, Tor}
    {\emph{Phonion is a Tor like anonymization network for Telephony or VoIP communications. Designed and implemented the Phonion architecture which offers better call quality than VoIP over Tor network and also supports offline calling.}}
  \entry
    {2014}
    {Android accessory driver}
    {Linux Kernel, Android accessory}
    {\emph{Developed a Kernel driver to setup any Android device into Accessory (ADK) mode. Also reused this as an independent module in developing a display driver for Linux on Android as part of Google Summer of Code.}}
  \entry
    {2012–2016}
    {MDroid - Moodle for Android}
    {Android, Githooks, REST APIs, Markdown}
    {\emph{Created a native open sourced Moodle application for Android. The app received over 80,000 downloads on Play Store and translated into 3 different languages by contributors.}}
%  \entry
%    {2014}
%    {Programmable Delay Logic}
%    {Verilog, Xilinx ISE, Virtex 5}
%    {\emph{Implemented PDL on virtex 5 FPGA. Configurable precision of 10 picoseconds.}}
\end{entrylist}

\end{document}
