% !TEX program = XeLaTeX
\documentclass[]{friggeri-cv}

\begin{document}
\header{Praveen Kumar }{Pendyala}{Android developer | Testability advocate | Clean code enthusiast}

% In the aside, each new line forces a line break
\begin{aside}
  \section{Information}
    \noindent\rule{3cm}{0.5pt}
    %DOB: 26.10.1992
    \href{http://pkp.io}{http://pkp.io}
    \href{https://github.com/praveendath92}{github/praveendath92}    
  \section{Contact}
    \href{mailto:mail@pkp.io}{mail@pkp.io}
    (+49) 176-833-55188
    Fulda, Germany
    ~
  \section{Skills}
    \noindent\rule{3cm}{0.5pt}
  \section{Programming}
    Java, Python, PHP, JavaScript, Matlab, Kernel C
  \section{Android}
    MVP design pattern, Dagger 2, Material Design, Retrofit
  \section{Testing}
    JUnit, Espresso, Mockito, Robolectric, Cucumber, Monkey tool, Selenium
  \section{Web}
    AngularJS, NodeJS, Npm, HTML5, REST APIs, Bootstrap, Jekyll
  \section{Tools}
    Git, Travis CI, Maven, Gradle, MySQL, \LaTeX
  \section{Platforms}
    Linux, Docker, Google App Engine, Heroku
  \section{MOOCs}
    Machine Learning (grade: 97.3\%)
    ~
    Algorithms: Design and Analysis, Part 1 (grade: 95.6\%)
\end{aside}
~
\section{Education}
\begin{entrylist}
  \entry
    {2015–2017}
    {Masters, {\normalfont Distributed Software Systems}}
    {Technische Universität Darmstadt}
    {\emph{GPA of 3.9 out of 4.0 in US grading -- converted from 1.55 in German grading.}}
  \entry
    {2010–2014}
    {Bachelors, {\normalfont Electrical Engineering}}
    {Indian Institute of Technology Bombay}
    {}%GPA: 7.49 / 10}
\end{entrylist}

\section{Experience}
\begin{entrylist}
  \entry
    {2014–2017}
    {Systems Security Lab, TU Darmstadt}
    {Android Developer and Security Researcher}
    {\emph{Successfully completed 3 different projects in Cloud security, Anonymous voice communications, and Analysis of attacks on Android. Co-authored 2 papers published in top tier conferences (one more in submission).}}
  \entry
    {05–08 2014}
    {Google Summer of Code}
    {Android Developer and Linux Kernel Programmer}
    {\emph{Developed kernel display drivers for accessing the GUI of a Linux device on Android over USB interface. Improved the frame rate from an initial 10fps to 25fps by employing partial updates for only modified areas of UI and by implementing Run-length Encoding for loss-less, inline compression of image updates.}}
\end{entrylist}

\section{Publications}
\begin{itemize}
  \item \textbf{Phonion: Practical Protection of Metadata in Telephony Networks} in \textit{Proceedings on Privacy Enhancing Technologies (PoPETs) 2017}
  \item \textbf{DroidAuditor: Forensic Analysis of Application-Layer Privilege Escalation Attacks on Android} in \textit{Financial Cryptography and Data Security Conference 2016}
\end{itemize}
\vspace{0.3cm}

\section{Achievements}
\begin{itemize}
  \item Winners of MAPPING (Managing Alternatives for Privacy, Property and Internet Governance) App Competition and \euro 20,000 prize money at CeBIT 2016, Hannover
  \item A. Richard Newton Young Student Fellow Award, DAC 2014, San Francisco
  %\item Recipient of Indian Association for Research in Computing Science grant, 2013
  %\item All India category Rank 1 among 470,000 students in IIT JEE 2010
\end{itemize}
\vspace{0.3cm}

\section{Projects}
\begin{entrylist}
  \entry
    {2015–2016}
    {Omnishare}
    {Android, Cryptography, Dropbox and Google Drive APIs}
    {\emph{Omnishare is an Android application for client side encryption of data before uploading to cloud storage services. Collaborated with a team of 5 in designing the encryption architecture. Developed the Android application using modular components and Factory design pattern for abstracting different cloud APIs.}}
  \entry
    {2015–2016}
    {Phonion}
    {Android, Python, Java, Twilio, Asterisk, Google Voice, Selenium}
    {\emph{Phonion is an anonymization network for voice communications. Designed and implemented the Phonion architecture which offers 50\% lower latency and 30\% better voice quality than alternatives like VoIP over Tor network. Implemented a distributed resource locking system for allocation of phone numbers from loosely co-operating entities.}}
  \entry
    {2012–2016}
    {MDroid - Moodle for Android}
    {Android, Githooks, REST APIs, Markdown}
    {\emph{Created and maintained a native open sourced Moodle application for Android. The application received over 80,000 downloads on Google Play Store and translated into 3 different languages by contributors.}}
\end{entrylist}

\end{document}
